\documentclass{udpreport}
\title{Creación de paquetes utilizando Scapy y validación con Wireshark}
\author{Integrantes: Thomas Muñoz, Ignacio Yanjari, Dagoberto Navarrete, Ignacio López.}
\date{Marzo de 2016}
\usepackage{graphicx}
\graphicspath{ {img/} }
\udpschool{Escuela de Informática y Telecomunicaciones}

\begin{document}
\maketitle
\tableofcontents
\chapter{Introducción}
	En este laboratorio buscamos aprender cómo se crean los paquetes con Scapy y como se organizan estos, además de entender la
	diferencia entre enviar un paquete a un equipo específico o a todos los pertenecientes a la red, a su vez entendimos la
	utilidad de Wireshark al tener que filtrar y capturar los paquetes que enviábamos entre nuestros equipos para verificar que se
	hayan recibido correctamente y estos no tuvieran algún tipo de problema.
\chapter{Actividades}
	\section{Instalacion de software}

	\section{Creacion de Paquetes}
		
	\section{Cuestionario}
	
	  1.-¿Qué pasa cuando envió un paquete a la dirección FF:FF:FF:FF:FF:FF? ¿Quienes
	     lo reciben? ¿Por qué?\\
	     
	  2.-¿Qué pasa cuando envió un paquete a una MAC de otro equipo? ¿Quieres lo
	      pueden reciben? ¿Por qué?\\
	      
	  3.-¿Qué sucede si envía un paquete a una MAC que no corresponda a ningún equipo
	      de la red? ¿Quienes lo pueden recepcionar? ¿Por qué?\\
	      

\chapter{Conclusión}
  
\begin{thebibliography}{x}

\end{thebibliography}
\end{document}
